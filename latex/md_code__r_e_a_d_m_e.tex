\subsection*{Índice}

{\bfseries 1.} Introdução {\bfseries 2.} Ficheiros do projeto {\bfseries 3.} Descrição da solução {\bfseries 4.} Manual de utilizador {\bfseries 5.} Conclusões e dificuldades {\bfseries 6.} Anexo / Referências

\subsection*{Introdução}

Neste projeto foi-\/nos pedido que fizéssemos um jogo em C que simulasse um cenário em que humanos (H) eram perseguidos por zombies (Z). Este deveria ser concebido numa grelha 2D tiroidal e utilizando dimensões X e Y e a vizinhança de Moore. Para além disto era pedido ainda que os agentes (humanos e zombies) fossem espalhados pela grelha sempre que o programa fosse iniciado, que fosse implementada uma IA (Inteligência Artificial) básica e que houvessem dois modos de jogo diferentes\+: um automático e um interativo.

\begin{quote}
{\bfseries Nota\+:} O jogo não foi finalizado. \end{quote}


\subsection*{Ficheiros}

\subsubsection*{Do projeto\+:}


\begin{DoxyItemize}
\item \mbox{\hyperlink{example_8c}{example.\+c}}
\item \mbox{\hyperlink{example_8h_source}{example.\+h}}
\item example.\+o
\item example
\item ini.\+c
\item ini.\+o
\item jogo.\+ini
\item moore.\+c
\item \mbox{\hyperlink{moore_8h_source}{moore.\+h}}
\item moore.\+o
\item movimento.\+c
\item \mbox{\hyperlink{movimento_8h_source}{movimento.\+h}}
\item movimento.\+o
\item \mbox{\hyperlink{showworld_8h}{showworld.\+h}}
\item \mbox{\hyperlink{showworld__simple_8c}{showworld\+\_\+simple.\+c}}
\item showworld\+\_\+simple.\+o
\item shuffle.\+c
\item \mbox{\hyperlink{shuffle_8h_source}{shuffle.\+h}}
\item shuffle.\+o
\item rect.\+png
\end{DoxyItemize}

\subsubsection*{Outros\+:}


\begin{DoxyItemize}
\item R\+E\+A\+D\+ME. md
\item Makefile
\item Doxyfile
\end{DoxyItemize}

\subsection*{Descrição da solução}

Relação entre ficheiros\+:


\begin{DoxyCode}
graph LR
B[shuffle] --> A[example]
C[movimento] --> A
D[moore] --> A
\end{DoxyCode}
 \begin{quote}
{\bfseries Nota\+:} O jogo não foi finalizado, por isso não foi possível chegar a uma solução. \end{quote}


\subsection*{Manual de utilizador}

{\bfseries 1.} Para criar uma {\itshape build} do jogo é necessário executar o seguinte comando no {\bfseries terminal}\+:

\$ make

{\bfseries 2.} Como jogar\+:

\begin{quote}
{\bfseries Nota\+:} Visto que o jogo não foi finalizado, não foi possível acabar o manual. \end{quote}


\subsection*{Conclusões e dificuldades}

O projeto embora ambicioso e interessante era bastante complexo. Como tal, apareceram bastantes obstáculos. Alguns foram ultrapassados mas muitos acabaram por se manter, dificultando assim a continuação do desenvolvimento do mesmo. No entanto, foram feitos progressos e muito foi aprendido, visto que o projeto nos permitiu treinar vários aspetos na utilização da linguagem C (como utilização de apontadores, {\itshape enums} e {\itshape structs}, embora os mesmos nem sempre tenham sido utilizados), do terminal de Ubuntu, nomeadamente na compilação de vários ficheiros, utilização de makefiles e navegação no mesmo.

\subsection*{Anexo / Referências}

{\bfseries Fontes utilizadas durante o desenvolvimento do projeto\+:}


\begin{DoxyItemize}
\item \href{https://www.tutorialspoint.com/cprogramming/c_structures.htm}{\tt https\+://www.\+tutorialspoint.\+com/cprogramming/c\+\_\+structures.\+htm}
\item \href{https://www.programiz.com/c-programming/c-pointers}{\tt https\+://www.\+programiz.\+com/c-\/programming/c-\/pointers}
\item \href{https://www.tutorialspoint.com/cprogramming/c_pointers.htm}{\tt https\+://www.\+tutorialspoint.\+com/cprogramming/c\+\_\+pointers.\+htm}
\item \href{https://boredzo.org/pointers/}{\tt https\+://boredzo.\+org/pointers/}
\item \href{https://stackoverflow.com/questions/37538/how-do-i-determine-the-size-of-my-array-in-c}{\tt https\+://stackoverflow.\+com/questions/37538/how-\/do-\/i-\/determine-\/the-\/size-\/of-\/my-\/array-\/in-\/c}
\item \href{https://stackoverflow.com/questions/6127503/shuffle-array-in-c}{\tt https\+://stackoverflow.\+com/questions/6127503/shuffle-\/array-\/in-\/c}
\item \href{https://stackoverflow.com/questions/42645455/shuffling-an-array-of-struct-in-c}{\tt https\+://stackoverflow.\+com/questions/42645455/shuffling-\/an-\/array-\/of-\/struct-\/in-\/c}
\item \href{https://stackoverflow.com/questions/34203053/shuffle-multidimensional-array-in-c}{\tt https\+://stackoverflow.\+com/questions/34203053/shuffle-\/multidimensional-\/array-\/in-\/c}
\item \href{https://stackoverflow.com/questions/2280352/use-an-ini-file-in-c-on-linux}{\tt https\+://stackoverflow.\+com/questions/2280352/use-\/an-\/ini-\/file-\/in-\/c-\/on-\/linux}
\item \href{http://www.ncbr.muni.cz/~martinp/g2/index.html}{\tt http\+://www.\+ncbr.\+muni.\+cz/$\sim$martinp/g2/index.\+html}
\end{DoxyItemize}

\subsection*{Metadados\+:}

{\bfseries Docente\+:} Nuno Fachada

{\bfseries Discente\+:} Rodrigo Garcia nº 21704304 {\bfseries Discente\+:} Tiago Alves nº 21701031

{\bfseries Curso\+:} Licenciatura em Aplicações Multimédia e Videojogos {\bfseries Instituição\+:} Universidade Lusófona de Humanidades e Tecnologias 